\section{Introduction}
\label{sec:intro}
Air quality models are valuable tools to investigate the complex and dynamic interactions between meteorology and chemistry leading to poor air quality episodes

\section{Methods}
\label{sec:methods}
  \subsection{Description of models}
  
 \begin{table} \begin{threeparttable}
    \caption{ HCHO yields from various species, and lifetime against oxidation by OH. }
    \begin{tabular}{  l  l  l  l  l  }
      \toprule
      Species           & HCHO Yield      & Life vs OH & Notes    & Source
      \\                & (molar \% )     &        &              &          \\
      \midrule
      Isoprene          & 315$\pm$50      &        & High NO$_X$  & a        \\
      & 285$\pm$30      &        & High NO$_X$  & a        \\
      & 225             & 35 min & High NO$_X$  & b        \\ % Done
      & 150             &        & Low  NO$_X$  & b        \\ % Done
      & 150             &        & Low  NO$_X$  & d        \\
      & 450             &        & High  NO$_X$ & d        \\
      $\alpha$-Pinene   & 28$\pm$3        &        & Low NO$_X$   & c        \\
      & X$\pm$3         &        & X NO$_X$     & d        \\
      & 230$\pm$90      &        & High NO$_X$  & a        \\
      & 190$\pm$50      &        & High NO$_X$  & a        \\
      & 19              & 1 hour &              & b        \\ % Done
      $\beta$-Pinene    & 65$\pm$6        &        & Low NO$_X$   & c      \\
      & X$\pm$3         &        & X NO$_X$     & d      \\
      & 540$\pm$50      &        & High NO$_X$  & a     \\
      & 450$\pm$80      &        & High NO$_X$  & a      \\
      & 45              & 40 min &              & b      \\ % Done
      Methane           & 100             & 1 year  &             & b     \\
      Ethane            & 180             & 10 days &             & b     \\
      Propane           & 60              & 2 days  &             & b     \\
      Methylbutanol     & .13(per C)      & 1 hour  &             & b     \\
      HCHO              & 100             & 2 hour  &             & b     \\
      Acetone           & .67(per C)      & 10 days &             & b     \\
      Methanol          & 100             & 2 days  &             & b     \\ %Done
      \bottomrule
    \end{tabular}
    \begin{tablenotes} % \item makes new lines
      \item a \citet{AtkinsonArey2003}: Table 2, Yield from Isoprene reaction with OH, two values are from two referenced papers therein.
      \item b \citet{Palmer2003}: lifetimes assume [OH] is 1e15 mol cm$^{-3}$.
      \item c \citep{Lee2006}: Calculated through change in concentration of parent and product linear least squares regression.
      Estimates assume 20$^\circ$~C conditions.
      \item d \citet{Wolfe2016}: ``prompt yield'': change in HCHO per change in ISOP$_0$.
      $[ISOP]_0=[ISOP]\exp(k_1[\mathrm{OH}]t)$; where $k_1$ is first order loss rate.
      Effectively relates HCHO abundance with isoprene emission strength
      \item e Calculated from satellite detected concentrations of HCHO.
      \item f Calculated using PTR-MS and iWAS on SENEX campaign data.
    \end{tablenotes}
    \label{ch_isop:tab:VOCLiteratureYields}
\end{threeparttable} \end{table}


  \subsection{Description of observations}
  
  \subsection{Statistical analyses}
   \paragraph{Ozone}
   
\begin {equation}
 NMSE = \frac{\Sigma_{i=1}^N (M_i - O_i)^2}{N x \overline{M} x \overline{O}}
\end {equation}  

where $\overline{M}$ is the average modeled value
   \paragraph{PM2.5}

\section{Model evaluation results}
\label{sec:results}
 
  \subsection{Ozone}
    \paragraph{Region/domain-wide analysis}
  
    \paragraph{Spatial analysis}
  \subsection{PM2.5}
   \paragraph{Region/domain-wide analysis}
  
  \paragraph{Spatial analysis}
  
 \subsection{PM2.5 speciation} 

\section{Discussion}
\label{sec:discussion}  


  \subsection{Installation} 
    If the document class \emph{elsarticle} is not available on your computer, you can download and install the system package \emph{texlive-publishers} (Linux) or install the \LaTeX\ package \emph{elsarticle} using the package manager of your \TeX\ installation, which is typically \TeX\ Live or Mik\TeX.

    Once the package is properly installed, you can use the document class \emph{elsarticle} to create a manuscript. Please make sure that your manuscript follows the guidelines in the Guide for Authors of the relevant journal. It is not necessary to typeset your manuscript in exactly the same way as an article, unless you are submitting to a camera-ready copy (CRC) journal.

    The Elsevier article class is based on the standard article class and supports almost all of the functionality of that class. In addition, it features commands and options to format the
    \begin{itemize}
      \item document style
      \item baselineskip
      \item front matter
      \item keywords and MSC codes
      \item theorems, definitions and proofs
      \item lables of enumerations
      \item citation style and labeling.
    \end{itemize}

    The author names and affiliations could be formatted in two ways:
    
    \begin{enumerate}[(1)]
      \item Group the authors per affiliation.
      \item Use footnotes to indicate the affiliations.
    \end{enumerate}
    
\section{Bibliography styles}

  There are various bibliography styles available. You can select the style of your choice in the preamble of this document. These styles are Elsevier styles based on standard styles like Harvard and Vancouver. Please use Bib\TeX\ to generate your bibliography and include DOIs whenever available.
  
  Here are two sample references: \cite{Feynman1963118,Dirac1953888}.
